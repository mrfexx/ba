% -*- TeX -*- -*- DE -*-

\chapter{Einleitung}
\label{ch:einleitung} Die Vorlage \lstinline!iiit-dipl.cls! ist die dritte Version der am IIIT f"ur
Studien- und Diplomarbeiten verwendeten Vorlage. Die aktuellen Revision wurde um einige packages
erweitert, die in den vielen Arbeiten in der einen oder anderen Form zu finden waren oder sich im
t"aglichen Gebrauch als hilfreich erwiesen haben. Es sollen hier auch kleine Beispiele gegeben
werden, um ein einheitliches Erscheinungsbild der am IIIT erstellten Arbeiten zu gew"ahrleisten.
Dabei wird ein gewisses Grundwissen "uber \LaTeX{} vorausgesetzt.

In Kapitel \ref{ch:einleitung} werden die technischen Grundlagen wie die eingebundenen packages
sowie die Dateistruktur dieser Beispielarbeit beleuchtet. Kapitel \ref{ch:modellierung} geht n"aher
auf die von den packages sowie der Klasse zur Verf"ugung gestellten Umgebungen ein.

\section{Dateistruktur}

Das zentrale Dokument einer Studien-/ Diplomarbeit ist \lstinline!diplom.tex!. Am Anfang dieser
Datei werden die grundlegenden Einstellungen an der Dokumentenklasse vorgenommen. Diese werden als
Argumente in eckigen Klammern an die Klasse "ubergeben.

Die Optionen der Dokumentenklasse k"onnen beispielsweise folgenderma"sen eingestellt werden:
\lstinputlisting[style=latex,firstline=2,lastline=31,firstnumber=2]{diplom.tex} Damit wird die Klasse
\lstinline!iiit-dipl! mit doppelseitigem Layout, alphanumerischem
Literaturverzeichnis und \lstinline!empheq! als Mathematikpaket ausgew"ahlt.

Einige der zur Verf"ugung stehenden switches sind in Tabelle \ref{tab:switches} aufgef"uhrt und erl"autert.
Defaultwerte sind mit eckigen Klammern markiert.

\newcolumntype{R}{>{\raggedleft\arraybackslash}X}
\begin{table}[tb]
    \centering
    \caption{Zentrale switches\label{tab:switches}}
    \begin{tabularx}{\textwidth}{>{\hsize=.6\hsize}R>{\hsize=1.4\hsize}X}
        \toprule
        switch          & Bedeutung\\
        \midrule
        \lstinline!latin1!, \lstinline!utf8!, \lstinline![ansinew]! & Zur direkten Eingabe von
                          Sonderzeichen wie "a, "o, "u und "s muss dem Compiler die verwendete
                          Kodierung mitgeteilt werden. \\
        \lstinline![deutsch]!, \lstinline!english! & Sprache der automatisch gesetzten Begriffe \\
        \lstinline!oneside!, \lstinline![twoside]! & Einstellung f"ur doppelseitigen Druck.
                          Beeinflusst die Kopfzeile und Leerseiten vor neuen Kapiteln. \\
        \lstinline!empheq!, \lstinline![amsmath]! & empheq ist eine Erweiterung der
                          Mathematik-packages amsmath. Es geh"ort aber nicht zur
                          Standardinstallation.\\
        \lstinline!biblatex! & Verwendet \lstinline!biblatex! statt \lstinline!bibtex!.
                          \lstinline!biblatex! hat den Vorteil, dass es die UTF8-Kodierung
                          unterst"utzt, die beispielsweise von JabRef verwendet wird. Damit k"onnen
                          beispielsweise auch Referenzen von Autoren mit Umlauten im Namen korrekt
                          sortiert werden. \lstinline!biblatex! erfordert, dass in der verwendeten
                          Umgebung \lstinline!biber.exe! als Bibtex-Compiler eingestellt ist. Falls
                          \lstinline!biber! nicht in der installierten \TeX-Distribution enthalten ist,
                          kann es nachtr�glich von
                          \url{http://sourceforge.net/projects/biblatex-biber/files/biblatex-biber/}
                          bezogen werden. Die Kompatibilit�ten der verschiedenen \lstinline!biblatex!
                          und \lstinline!biber! Versionen k�nnen der \lstinline!biblatex! Dokumentation
                          \cite{biblatex} entnommen werden. \\
        \lstinline!lst! & Sollte angegeben werden, wenn Quelltexte eingebunden werden. Mit dieser
                          Option wird das Listings-Package geladen und es werden Styles mit
                          Syntax-Highlighting f"ur verschiedene Programmiersprachen (\lstinline!C++!,
                          \lstinline!matlab!, \lstinline!latex!, \lstinline!other!) vordefiniert. Wird
                          das Listings-Package nicht ben"otigt, empfiehlt es sich, diese Option
                          wegzulassen, um die Compiliergeschwindigkeit zu erh"ohen.\\
        andere          & Andere Optionen werden von der \lstinline!scrbook!-Klasse behandelt.\\
        \bottomrule
    \end{tabularx}
\end{table}

Um die Titelseite korrekt generieren zu k"onnen, m"ussen einige (selbsterkl"arende) Variablen gesetzt
werden. Dabei ergeben die folgenden Einstellungen die Titelseite dieses Dokuments:
\lstinputlisting[style=latex,firstline=35,lastline=41,firstnumber=35]{diplom.tex}
Danach kann mit dem eigentlichen Dokument begonnen werden. Zuerst wird die frontmatter, also das
vorbereitende Material pr"asentiert. Hierzu geh"oren die Titelseite mit Einverst"andniserkl"arung
(\lstinline!\maketitle!) und das Inhaltsverzeichnis (\lstinline!\tableofcontents!), die automatisch
generiert werden, sowie der Abstract, eine ca. 100-200 Wort lange Zusammenfassung der Arbeit
inklusive der erzielten Ergebnisse auf Englisch.

Im Hauptteil (\lstinline!\mainmatter!) wird die eigentliche Arbeit pr"asentiert. Der Hauptteil
sollte die folgenden Teile beinhalten:
\begin{itemize}
      \item
            Einleitung
      \item
            Zusammenfassung und Ausblick
\end{itemize}
Es ist sinnvoll, den Hauptteil in mehrere Dateien aufzuteilen. Zus"atzliche Dateien k"onnen mit Hilfe des
\lstinline!\include!-Befehls eingebunden werden. Dabei wird der Inhalt der Datei so interpretiert, als
st"unde er direkt in der Hauptdatei.

Der letzte Teil des Hauptteils sind die Anh"ange. In ihnen werden z.B. Variablen, wichtige Teile des
Quellcodes oder l"angere Herleitungen beschrieben. Nach dem \lstinline!\appendix!-Befehl werden die
Anh"ange automatisch alphabetisch nummeriert.

Am Abschluss des Dokuments steht das wichtige Literaturverzeichnis sowie ggf. das Bilder- und
Tabellenverzeichnis.

\lstinputlisting[style=latex,firstline=55,lastline=56,,firstnumber=55]{diplom.tex}

Da die Vorlage, also \lstinline!iiit-dipl.cls! \textit{nicht} ver"andert werden soll, steht f"ur
eigene Erweiterungen wie zum Beispiel neue n"utzliche packages die Datei
\lstinline!erweiterungen.tex! zur Verf"ugung, die in die Hauptdatei eingebunden wird.
\lstinputlisting[style=latex,firstline=32,lastline=32,,firstnumber=32]{diplom.tex}
In ihr wird standardm"a"sig das Verzeichnis f"ur die Bilder gesetzt sowie ein Beispiel f"ur eine
erweiterte Trennregel gegeben. Zweck dieser Datei ist es, alle Anpassungen, die die aktuelle Arbeit
ben"otigt, zentral zu speichern und damit unabh"anig von der eigentlichen Diplomklasse zu machen.

\lstinputlisting[style=latex]{erweiterungen.tex}

\section{Allgemeine Hinweise}

Im Laufe der Zeit haben sich verschiedene Vorgehensweisen bei der Erstellung einer Diplom- bzw.
Studienarbeit als n"utzlich erwiesen. Einige von ihnen sollen hier ohne Anspruch auf Vollst"andigkeit
aufgef"uhrt werden.

\begin{description}
      \item[Fr"uh dokumentieren]
            Es macht Sinn, auch Zwischenergebnisse ggf. auch in \LaTeX{} zu dokumentieren.

      \item[Sinnvolle Dateigr"o"sen]
            \LaTeX{} bietet die M"oglichkeit, die Arbeit auf mehrere Dateien aufzuteilen und diese in die
            Hauptdatei einzuf"ugen.
\end{description}

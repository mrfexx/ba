% -*- TeX -*- -*- DE -*-

\chapter{Modellierung}
\label{ch:modellierung}
\section{Ein Abschnitt}

\begin{definition}[Operator]
      \dots{}
\end{definition}

\begin{satz}[Mittelwertsatz der Differentialrechnung]
  Mit \cite{bronstein:12} folgt \dots{}
\end{satz}

\begin{beispiel}[Berechnung der Fouriertransformation]
  ist trivial.
\end{beispiel}

\begin{bemerkung}
      Was ich schon immer mal sagen wollte.
\end{bemerkung}

\begin{definition}[thmmarks manuell setzen]
      Zur besseren Positionierung der thmmarks k"onnen diese auch manuell gesetzt werden.

      \begin{itemize}
            \item Zum Beispiel bei Aufz"ahlungen. \thmmark
      \end{itemize}
\end{definition}


Das Institutslogo ist in Bild \ref{fig:iiitlogo} zu sehen.

%\LoadBild[width=20mm]{iiit}{Das IIIT-Logo\label{fig:iiitlogo}}
\begin{figure}[ht]
    \centering
    \includegraphics[width=20mm]{iiit}
    \caption{Das IIIT-Logo.\label{fig:iiitlogo}}
\end{figure}

Wenn mehrere Bilder im Zusammenhang stehen, k"onnen sie gemeinsam angegeben und mit \ref{subfig:iiitlogo} und \ref{subfig:kitlogo} referenziert werden.
\begin{figure}[tb]
	\centering
	\null
	\hfill
	\begin{subfigure}{.3\textwidth}
		\centering
		\includegraphics[height=20mm]{iiit}
		\caption{Das IIIT-Logo.}
		\label{subfig:iiitlogo}
	\end{subfigure}
	\hfill
	\begin{subfigure}{.3\textwidth}
		\centering
		\includegraphics[height=20mm]{kit_logo_de_1c_schwarz}
		\caption{Das KIT-Logo.}
		\label{subfig:kitlogo}
	\end{subfigure}
	\hfill\null % \null fuegt unsichtbares Zeichen ein, damit \hfill funktioniert. Zum Verstaendnis einfach mal \null auskommentieren und sehen, was passiert.
	\caption[Logos des Instituts und der Universit"at.]{Logos des Instituts \subref{subfig:iiitlogo} und der Universit"at \subref{subfig:kitlogo}.}
	\label{fig:Logos}
\end{figure}

\section{Zahlen mit Zehnerpotenz}
Die Trennung zur Zehnerpotenz erfolgt durch \verb|\cdot|.
\begin{verbatim}
 \num{1.234e6}
\end{verbatim}
\begin{align}
 \num{1.234e6}
\end{align}

\section{Einheiten \SI{5}{\um}}

Einheiten werden gerade gesetzt: $\SI{1}{m}$.
\begin{verbatim}
 \SI{1}{m\per s} = \SI{3,6}{\km\per h}
\end{verbatim}
\begin{align}
      \SI{1}{m\per s} = \SI{3,6}{\km\per h}
\end{align}

\section{Vektoren}

Vektoren und Matrizen sollten als solche gekennzeichnet werden (aufrecht und fett)
\begin{verbatim}
 \vec{x} = \vec{A}\vec{\Phi}\vec{y}
\end{verbatim}

\begin{align}
 \vec{x} = \vec{A}\vec{\Phi}\vec{y}
\end{align}

\section{Macros}
\paragraph{Vektoren und Matrizen werden fett gedruckt}
$$\vec{x}$$

\paragraph{Stochastische Gr��en in Typewriter}
$$\stoch{x}$$


\paragraph{Zahlenmengen}
$$\rz$$
$$\nz$$
$$\gz$$
$$\cz$$
$$\qz$$

\paragraph{Indizes f�r Eingang- und Ausgang}
$$x_{\ee}$$
$$x_{\EE}$$
$$x_{\aaa}$$
$$x_{\AAA}$$

\paragraph{Indizes f�r Abtastung}
$$f_\abtast$$

\paragraph{Indices die keiner nummerierung entsprechen}
$$x_\namingIndex{a}$$

\paragraph{Differentiale}
$$\ddd$$

\paragraph{Einheitsmatrix}
$$\II$$

\paragraph{Erwartungswert}
$$\E{x}$$

\paragraph{Gr��ten gemeinsame Teile und kleinster gemeineinsames Vielfaches}
$$\ggT$$
$$\kgV$$

\paragraph{imagin�re Einheit}
$$\jj$$
$$\ii$$

\paragraph{Fehlende Funktionsnamen}
$$\Res$$
$$\sinc$$
$$\sign$$

\paragraph{Transponieren einer Matrix}
$$\TT$$
$$\vec{A}^\TT$$

\paragraph{Macros f�r h�ufige mathematische Ausdr�cke}
$$\eexp{x}$$
$$\dd$$
$$\sbe$$
$$\entspr$$
$$\pow{x}$$
$$\norm{x}$$
$$\abs{x}$$
$$\infint$$
$$\vr{x}{y}$$
$$\InP{x}{y}$$
$$\TZ$$
$$\ZT$$
$$\vTZ$$
$$\vZT$$
$$\uint{x}$$


\paragraph{Transformationen}
$$\FT$$
$$\iFT$$


\paragraph{Real und Imagin�rteil}
$$\Real$$
$$\Imag$$

\paragraph{Schriftgr��e in Formeln}
$$\T \int_{0}^{\infty}{a^2+b^2=c^2}$$ 
$$\D \int_{0}^{\infty}{a^2+b^2=c^2}$$

\section{Tabellen}

Tabelle~\ref{tab:booktabs} zeigt die Verwendung des Booktabs-Packages.

\begin{table}[ht]
      \centering
      \caption{Booktabs\label{tab:booktabs}}
      \begin{tabular}{cc}
            \toprule
                  Meisen & andere V"ogel\\
            \midrule
                  A-Meise & Amsel\\
                  B-Meise & Drossel\\
                  C-Meise & Fink\\
                  D-Meise & Star\\
            \bottomrule
      \end{tabular}
\end{table}

The introduced model allows to estimate the temperature at different points in the patient's body and on its surface. To evaluate the accuracy the arterial, the venous and the nasal temperature of the patient as well as the room temperature has been recorded during cardiac surgeries and compared to the results of the simulation. Although external sources of heat have not been taken into account so far, the occurring deviations were less than

\lstinputlisting[caption={Anderer Quelltext},style=matlab]{haar_zerlegung.m}

The introduced model allows to estimate the temperature at different points in the patient's body and on its surface. To evaluate the accuracy the arterial, the venous and the nasal temperature of the patient as well as the room temperature has been recorded during cardiac surgeries and compared to the results of the simulation. Although external sources of heat have not been taken into account so far, the occurring deviations were less than \lstinline!plot(x,y)! The introduced model allows to estimate the temperature at different points in the patient's body and on its surface. To evaluate the accuracy the arterial, the venous and the nasal temperature of the patient as well as the room temperature has been recorded during cardiac surgeries and compared to the results of the simulation. Although external sources of heat have not been taken into account so far, the occurring deviations were less than

The introduced model allows to estimate the temperature at different points in the patient's body and on its surface. To evaluate the accuracy the arterial, the venous and the nasal temperature of the patient as well as the room temperature has been recorded during cardiac surgeries and compared to the results of the simulation. Although external sources of heat have not been taken into account so far, the occurring deviations were less than

The introduced model allows to estimate the temperature at different points in the patient's body and on its surface. To evaluate the accuracy the arterial, the venous and the nasal temperature of the patient as well as the room temperature has been recorded during cardiac surgeries and compared to the results of the simulation. Although external sources of heat have not been taken into account so far, the occurring deviations were less than

The introduced model allows to estimate the temperature at different points in the patient's body and on its surface. To evaluate the accuracy the arterial, the venous and the nasal temperature of the patient as well as the room temperature has been recorded during cardiac surgeries and compared to the results of the simulation. Although external sources of heat have not been taken into account so far, the occurring deviations were less than

The introduced model allows to estimate the temperature at different points in the patient's body and on its surface. To evaluate the accuracy the arterial, the venous and the nasal temperature of the patient as well as the room temperature has been recorded during cardiac surgeries and compared to the results of the simulation. Although external sources of heat have not been taken into account so far, the occurring deviations were less than

The introduced model allows to estimate the temperature at different points in the patient's body and on its surface. To evaluate the accuracy the arterial, the venous and the nasal temperature of the patient as well as the room temperature has been recorded during cardiac surgeries and compared to the results of the simulation. Although external sources of heat have not been taken into account so far, the occurring deviations were less than The introduced model allows to estimate the
temperature at different points in the patient's body and on its surface. To evaluate the accuracy the arterial, the venous and the nasal temperature of the patient as well as the room temperature has been recorded during cardiac surgeries and compared to the results of the simulation. Although external sources of heat have not been taken into account so far, the occurring deviations were less thanThe introduced model allows to estimate the temperature at different points in the patient's body and on its surface. To evaluate the accuracy the arterial, the venous and the nasal temperature of the patient as well as the room temperature has been recorded during cardiac surgeries and compared to the results of the simulation. Although external sources of heat have not been taken into account so far, the occurring deviations were less than

\lstinputlisting[caption={Quelltext},style=C++]{interpretCDImessage.c}

The introduced model allows to estimate the temperature at different points in the patient's body and on its surface. To evaluate the accuracy the arterial, the venous and the nasal temperature of the patient as well as the room temperature has been recorded during cardiac surgeries and compared to the results of the simulation. Although external sources of heat have not been taken into account so far, the occurring deviations were less than The introduced model allows to estimate the temperature at different points in the patient's body and on its surface. To evaluate the accuracy the arterial, the venous and the nasal temperature of the patient as well as the room temperature has been recorded during cardiac surgeries and compared to the results of the simulation. Although external sources of heat have not been taken into account so far, the occurring deviations were less than

The introduced model allows to estimate the temperature at different points in the patient's body and on its surface. To evaluate the accuracy the arterial, the venous and the nasal temperature of the patient as well as the room temperature has been recorded during cardiac surgeries and compared to the results of the simulation. Although external sources of heat have not been taken into account so far, the occurring deviations were less than\cite{trees:68}

\begin{align*}
      \cmatrix{1 & 2\\3 & 4}
\end{align*}
